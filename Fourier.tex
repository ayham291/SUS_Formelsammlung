\section{Signaldarstellung im Frequenz- und Bildbereich}
\subsection{Harmonische Synthese}
Die Überlagerung von Sinusschwingungen zu einem periodischen,
nichtsinusförmigen Signal nennt man harmonische Synthese.
\subsubsection{Reelle Fourierreihe}
\begin{mdframed}[style=exercise]
    \begin{itemize}
        \item mit $sin$ und $cos$:
            \[
                f(t) = a_0 \sum_{k=1}^\infty [a_k\cdot cos(k\omega_1 t) b_k\cdot sin(k\omega_1 t)]
            \]
        \item mit Amplitude und Phase:
            \begin{align*}
                f(t) &= A_0+\sum_{k=1}^\infty [A_k\cdot cos(k\omega_1 t + \varphi_k)]\\
                    &= A_0+\sum_{k=1}^\infty [A_k\cdot sin(k\omega_1 t + \varphi_k-\frac{\pi}{2})]
            \end{align*}

            \texttt{Koeffizienten }
            \[
                A_0 = \frac{1}{T}\int_{t_0}^{T+t_0} f(t)dt
            \]
            \begin{align*}
                a_k = \frac{2}{T}\int_{t_0}^{T+t_0} f(t)\cdot cos(k\omega_1 t) dt\\
                b_k = \frac{2}{T}\int_{t_0}^{T+t_0} f(t)\cdot sin(k\omega_1 t) dt
            \end{align*}
    \end{itemize}
\end{mdframed}

\subsubsection{Komplexe Fourierreihe}
\begin{mdframed}[style=exercise]
    \[
        f(t)=\sum_{k=-\infty}^{\infty} \underline{c}_k\cdot e^{j\omega_1 k t}
    \]
    \begin{align*}
        \underline{c}_k &= \frac{1}{T}\int_{t_0}^{T+t_0} f(t)\cdot e^{-j\omega_1 k t}dt
                        &= \frac{1}{2}\left( a_k-jb_k \right)
    \end{align*}
\end{mdframed}

\subsubsection{Symmetrieeigenschaften}
\begin{itemize}
    \item Gerade Funktionen
        symmetrisch zur y-Achse\\
        alle $sin$-teile verschwinden

        - $A_0 = \frac{2}{T}\int^{\frac{T}{2}}_{0} y(t)dt$\\
        - $a_{k} = \frac{4}{T}\int^{\frac{T}{2}}_{0}y(t)\cdot cos(k\omega_1t)dt$\\
        - $b_k = 0$\\
    \item Ungerade Funktionen
        symmetrisch zum Ursprung\\
        alle $cos$-teile und Gleichanteil verschwinden

        - $A_0 = 0$\\
        - $a_k = 0$\\
        - $b_{k} = \frac{4}{T}\int^{\frac{T}{2}}_{0} y(t)\cdot sin(k\omega_1t)dt$\\
\end{itemize}
\subsubsection{Halbwellensymmetrie}
Halbwellensymmetrie gilt wenn:
\[
    y(t) = -y(t \pm T/2)
\]
Die Fourier-Reihe einer Zeitfunktion mit HWS enthält stets
nur Terme mit ungeraden Ordnungszahlen. $k=1,3,5,\dots,\infty$
\begin{mdframed}[style=exercise,frametitle=im Allgemeinen]
    \texttt{Koeffizienten}:\\
    \[
        A_0 = 0,\
        a_{2k} = 0,\
        b_{2k} = 0
    \]
        $$a_{2k-1} = \frac{4}{T}\int^{\frac{T}{2}}_{0}y(t)\cdot cos((2k-1)\omega_1t)dt$$
        $$b_{2k-1} = \frac{4}{T}\int^{\frac{T}{2}}_{0}y(t)\cdot sin((2k-1)\omega_1t)dt$$
\end{mdframed}
\begin{mdframed}[style=exercise,frametitle=gerade Halbwellensymmetrie]
    \[
        A_0 = 0,\
        b_k = 0,\
        a_{2k} = 0
    \]
    $$a_{2k-1} = \frac{8}{T}\int^{\frac{T}{4}}_{0}y(t)\cdot cos((2k-1)\omega_1t)dt$$
\end{mdframed}
\begin{mdframed}[style=exercise,frametitle=ungerade Halbwellensymmetrie]
    \[
        A_0 = 0,\
        a_k = 0,\
        b_{2k} = 0
    \]
    $$b_{2k-1} = \frac{8}{T}\int^{\frac{T}{4}}_{0}y(t)\cdot sin((2k-1)\omega_1t)dt$$
\end{mdframed}
\subsection{Verschiebungssatz}
